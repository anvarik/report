\documentclass{article}
\bibliographystyle{plain}
\usepackage{url}
\usepackage[pdftex,colorlinks=true,linkcolor=black,citecolor=blue,%
 anchorcolor=black,urlcolor=black,bookmarks=true,%
 bookmarksopen=true,bookmarksopenlevel=0,plainpages=false%
 bookmarksnumbered=true,hyperindex=false,pdfstartview=%
 ]{hyperref}
\begin{document}
\title{Survey on Databases being used in Electronic Health Records}   % type title between braces
\author{Enver Bahar}         % type author(s) between braces
\date{\today}    % type date between braces
\maketitle

\begin{abstract}
This article gives a brief comparison between SQL and NoSQL databases. It discusses what kind of database specifications might be beneficial for storage of Electronic Health Records (EHR), and gives examples of EHR systems using both SQL and NoSQL solutions. Finally it gives some suggestions for internal use cases. \\
\end{abstract} 

Storage of data and its process are the key points in most of the business cases.  Healtcare industry which also deals with a lot of patient information is one of the key industries in this field. Healthcare information is not just about number of records or entities. It deals with hugely diversified data from different specialties (like cardialogy, neurology, etc..), and need to deal with data of Clinical Care, Medication, Labs, History, Demographics, Reports and more. And we should clearly understand that each and every data will have changes every now and then. For example, healthcare standards organization may ask any healthcare providers to introduce / modify / remove a procedure in any given specialty any time. Hence the platform design should incorporate all these demands without any compromise\cite{online3}. And maybe the most important point is that healtcare information is extremely conservative: people could die if the data is inconsistent\cite{online4}.
 
 % sounds lame rewrite first sentence
Electronic Health Records are really important and getting more and more popular. Currently a lot of hospitals are trying to store EHR so that extensive queries can be run against and speed up the process of information retrieval about the storage systems. Even governments are spending billions of dollars to encourage doctors and hospitals to switch to electronic records to track patient care\cite{online5}.

\section{SQL vs NoSQL - which is better for EHR}

There are several database systems available, and making a decision between available options is a pretty tough issue and mostly depends on the use cases. Beforehand a database selection, a lot of questions need to be answered such as: How big is the database? How will the database grow over time? What are the required access speeds? Should data be partitioned? In this section we will explain two main database branches, I will provide their pros and cons and try to compare them according to these questions.

Let's first talk about the SQL databases which are also called Relational Databases. 

\subsection{SQL - Relational Databases}
SQL(Structured Query Language) databases are formally named as \emph{Relational Database}. Because it is manipulated by SQL language, the term "SQL Database" became more popular. Relational database is perceived by the user as a collection of two dimensional tables where all data are stored. The concept is proposed by Dr. Codd in 1970 and since 1980 it is the predominant choice for the storage of information in new databases used for financial records, manufacturing and logistical information, personnel data, and much more \cite{online1}. Basically relational model consists of components such collection of objects and relations which are stored as tables and set of operations to act on the relations.

Relational databases are managed by Relational Database Management System(RDBMS). RDBMS is basically a software to manage the database.  According to DB-Engine \cite{online2}, which is an initiative to collect and present information on database management systems, top five RDBMS by means of market sharing are as following: Oracle, MySQL, Microsoft SQL Server(MSSQL), PostreSQL and DB2. In Brainlab, all of the databases are Relational Databases and the database management system that is currently being used is MSSQL.

Let's talk about the characteristics of relational databases. First of all relational databases have a long history, which means that they are stable, and have long established standard. All relational databases offer SQL as a language, therefore moving from one vendor to another is trivial. 

\subsection{NoSQL Databases}
%http://www.mongodb.com/nosql
%There are arguments against NoSQL that aren't simply defenses of SQL. However, the early pioneers of NoSQL (Google and Amazon especially) developed databases after determining that relational databases didn’t serve their needs—setting the ground for the direct assault on SQL.

% Creating an ER diagram, normalizing tables, choosing primary keys, and setting relationships are all so ingrained in RDBMS users that it is difficult to separate the data modeling process from the underlying goal of data management.   When one has spent so much time building and understanding relational schema, moving to a schema-less data store feels completely wrong.

\section{Market Examples}
% having sparse matrixes, use nosql
% query by key
% easy to store
%As a simple example, you might have a Contacts database, and each person in the database could have varying fields. Some contacts might have several phone numbers of different types; some contacts might have several email addresses. Yes, you could normalize this into a relational database. But in this imagined application, rarely would you just get a list of phone numbers. Instead, most times you would want a name with all phone numbers associated with that name. Is it really necessary to force normalization into the picture and always store the phone numbers in a separate table, only to be joined during virtually every single lookup?
%You can certainly do it either way: Store the phones in a separate phone table. Or you could use document storage and just group it all together, and then pull the data in as a single object. This is where a database such as MongoDB really shines. You can store ad-hoc records, and pull them into an object in one shot. You can index the collection on any fields you want (most likely name) and then pull the document into a single JavaScript object with one lookup.

\section{Which one?}
Adding a new information about a patient let's say tumor case. In SQL this needs to result as a new table column addition. Which needs to be run to the whole table in the database. However in NoSQL document based database this can be done easily with simple add statement. So the user can go and add. 


\bibliography{bibliography/literature}
\end{document}